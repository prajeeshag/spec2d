\documentclass{article}
\usepackage[utf8]{inputenc}

\title{AMFI Technical Guide}
\author{prajeesh a g }
\date{September 2018}

\usepackage{natbib}
\usepackage{graphicx}
\usepackage{listings}
\usepackage{hyperref}

\begin{document}

\maketitle

\section{Introduction}
AMFI (Atmospheric Model From IITM) is spectral global atmospheric model..... blah blah....
\section{Dependencies}
Required external libraries:
\begin{itemize}
    \item MPI library(version $>=$ 3)
    \item Netcdf (version $>=$ 4)
    \item lapack (Math Kernel Libraries)
    \item NCL (version $>$ 6.2.0)
\end{itemize}
\section{Downloading the source code and input data}
The source code of AMFI is available through git repository. To download the code run the following command from your terminal.

\textbf{git clone \url{https://prajeeshag@bitbucket.org/prajeeshag/spec2d.git}}

This will create a directory named "spec2d". [NOTE: You can rename (or mv) this to any other name you want; henceforth referred as ROOTDIR]\newline
Change directory to ROOTDIR and execute \textbf{./init.sh}. [NOTE: \textbf{init.sh} should be run whenever the ROOTDIR is moved to a new location, or any new updates are pulled from git repository.]\newline
\textbf{cd spec2d} \newline
\textbf{./init.sh} \newline
Download the input data by running:
\textbf{./scripts/data\_download.sh}


\section{Compiling}
Edit ROOTDIR\textbf{/scripts/make\_spec.sh} to specify the compilers, compiler options and library paths etc.\newline
The Following are required fields to be set in ROOTDIR/scripts/make\_spec.sh.
\begin{itemize}
    \item \textbf{FC, F77 - Fortran compilers} (eg. mpiifort, mpif90, etc)
    \item \textbf{CC, MPICC - C compiler} (eg. mpiicc, mpiicc, etc)
    \item \textbf{NETCDF - Netcdf library path} 
    \item \textbf{FFLAGS - Fortran compiler options}
    \item \textbf{CFLAGS - C compiler options}
    \item \textbf{LDFLAGS - Linker options}
    \item \textbf{MPI3} - set to True if \textbf{MPI library version is $>=$ 3} or else set to False. [NOTE: While compiling if you get an error saying something about \textbf{missing MPI\_IALLREDUCE} then set this option to False]
    \item \textbf{AQUAPLANET} - if set True, AQUAPLANET configuration will be compiled.
\end{itemize}
Execute ROOTDIR/\textbf{scripts/make\_spec.sh} to compile the model.\newline
A successful compilation should produce following executables in ROOTDIR/exec directory.
\begin{itemize}
    \item \textbf{exec/amfi\_grid/amfi\_grid} - For generating AMFI P-grid (PACKED grid)
    \item \textbf{exec/p2r\_xgrid/p2r\_xgrid} - For generating AMFI P-grid to LATLON grid mapping grid.
    \item \textbf{exec/run\_mppnccp2r/run\_mppnccp2r} - For combining the parallel I/O output files and unpacking it from P-grid to LATLON grid.
    \item \textbf{exec/xgrid/xgrid} - For generating grid\_spec.nc
    \item \textbf{exec/spec2d/spec2d.exe} - Model executable
\end{itemize}

\section{Creating an Experiment}
Create a new directory for the experiment [NOTE: henceforth refered as \textbf{EXPDIR}].\newline Change directory to EXPDIR \newline
copy ROOTDIR/\textbf{scripts/prepare\_exp\_from\_scratch.sh} to EXPDIR \newline
set \textbf{NLAT} (number of latitudes) in prepare\_exp\_from\_scratch.sh file\newline
run prepare\_exp\_from\_scratch.sh [Note: this might take a while to complete depending upon the NLAT, set AQUAPLANET=True if a aquaplanet set up is needed]\newline

A successful execution of prepare\_exp\_from\_scratch.sh should produce the following files and folders in EXPDIR.

\begin{itemize}
    \item \textbf{data\_table} - forcing and other input data are specified here
    \item \textbf{field\_table} - tracer details are given here
    \item \textbf{input.nml} - namelist options
    \item \textbf{RESTART} - empty directory
    \item \textbf{diag\_table} - model outputs are specified here
    \item \textbf{INPUT} - Input directory
    \item \textbf{run\_amfi\_lsf.sh} - job submission script template (lsf job scheduler)
\end{itemize}
And the \textbf{INPUT} directory will contain the following files (\textbf{Not applicable for Aquaplanet setup!}).
\begin{itemize}
    \item \textbf{atm.res} - time stamp file where starting date of model and calendar type are given.
    \item albedo.nc 
    \item emis\_ref.nc - Land reference emisivity 
    \item ozone.nc           
    \item slopetype.nc    
    \item temp\_pres.nc - Atmospheric initial condition for cold start.
    \item vegfrac.nc
    \item amfi\_grid.nc
    \item grid\_spec.nc  
    \item p\_xgrd.nc           
    \item soiltype.nc   
    \item tg3.nc         
    \item vegtype.nc
    \item mtn.nc
    \item sea\_ice\_forcing.nc - Sea-Ice forcing 
    \item sst\_forcing.nc - SST forcing
    \item topography.nc
    \item zorl.nc - Land roughness
\end{itemize}

\section{Running the model}
The following parameters should be set accordingly in input.nml before running the model.
\begin{itemize}
    \item \textbf{layout - [npes\_y,npes\_x]} processor layout. npes\_y is number of decomposition in y-direction and npes\_x is number of decomposition in x-direction. All valid npes\_x and npes\_y values for a particular number of latitude (NLAT) and spectral truncation (TRUNC) will be given in a file valid\_pe\_layouts\_NLAT\_TRUNC.
\end{itemize}

Once the above parameters are set, execute \textbf{./run\_amfi\_lsf.sh} to run the model. 
If LSF job scheduler is available \textbf{run\_amfi\_lsf.sh} will submit the model run along with the post-processing run to job scheduler. In that case one should provide proper "\textbf{queue name}", "\textbf{wall clock time}", etc in run\_amfi\_lsf.sh. For any other job scheduler this script should be modified accordingly.

If no job scheduler is available run\_amfi\_lsf.sh will submit the run as usual on bash shell.

run\_amfi\_lsf.sh can also be used to submit the combining and P-grid to regular grid post-processing job. These are controlled by the following options inside run\_amfi\_lsf.sh:
\begin{itemize}
	\item \textbf{submit\_combine} - if set to "False" do not submit postprocessing, if set to "True" submit postprocessing
	\item \textbf{nproc\_combine} - number of processors for postprocessing 
	\item \textbf{combine\_only} - if "True" submit postprocessing only
	\item \textbf{combine\_child\_run} - if "1" postprocessing is submitted as child run of model run, if "0" then submit as an independent run.
\end{itemize}


\end{document}
